	\documentclass[12pt]{article}
		\usepackage{a4}
		\usepackage{makeidx}
		\title{wxwrapper Library Index}
		\date{\today}
		\def\mtl{{\tt\~\relax}}
		\author{Generated with KDOC Revision: 1.14 }
		\makeindex
		\begin{document}
		\maketitle
		\pagestyle{headings}
		\tableofcontents
		\newpage
		\sloppy
\section[wxView3D]{\emph{wxView3D} class reference \small\sf (wxView3D.h)}
\vspace{-0.6cm}
\hrulefill
\index{wxView3D}
\begin{description}
\item{Inherits:} wxGLCanvas
 wxBOOGAView
\end{description}
\begin{description}
\item[Description:] The wxView3D class can be used to display a World3D with a
GLRenderer 
\end{description}
The wxView3D class can be used to display a World3D with a
GLRenderer. The wxView msu be attached to a ViewManager.
wxViewCommands can be used to execute actions with the mouse.


\subsection{public members}
\subsubsection*{wxView3D::wxView3D}
\vspace{-0.65cm}
\hrulefill
\index{wxView3D!wxView3D}
\index{wxView3D!wxView3D}
\begin{flushleft}
\texttt{wxView3D(wxWindow *parent, int x, int y, int w, int h,              long style=wxRETAINED, char *name="test");}
\end{flushleft}

parent is the parent pointer to the Window.


\subsubsection*{wxView3D::\mtl wxView3D}
\vspace{-0.65cm}
\hrulefill
\index{wxView3D!\mtl wxView3D}
\index{\mtl wxView3D!wxView3D}
\begin{flushleft}
\texttt{\mtl wxView3D();}
\end{flushleft}



\subsubsection*{wxView3D::OnPaint}
\vspace{-0.65cm}
\hrulefill
\index{wxView3D!OnPaint}
\index{OnPaint!wxView3D}
\begin{flushleft}
\texttt{void OnPaint(void);}
\end{flushleft}



\subsubsection*{wxView3D::OnSize}
\vspace{-0.65cm}
\hrulefill
\index{wxView3D!OnSize}
\index{OnSize!wxView3D}
\begin{flushleft}
\texttt{void OnSize(int w, int h);}
\end{flushleft}



\subsubsection*{wxView3D::OnEvent}
\vspace{-0.65cm}
\hrulefill
\index{wxView3D!OnEvent}
\index{OnEvent!wxView3D}
\begin{flushleft}
\texttt{void OnEvent(wxMouseEvent\& event);}
\end{flushleft}



\subsubsection*{wxView3D::renderSelection}
\vspace{-0.65cm}
\hrulefill
\index{wxView3D!renderSelection}
\index{renderSelection!wxView3D}
\begin{flushleft}
\texttt{void renderSelection();}
\end{flushleft}

This method is used internally to render the boundingbox of selected objects.


\subsubsection*{wxView3D::useRenderingQuality}
\vspace{-0.65cm}
\hrulefill
\index{wxView3D!useRenderingQuality}
\index{useRenderingQuality!wxView3D}
\begin{flushleft}
\texttt{void useRenderingQuality();}
\end{flushleft}



\subsubsection*{wxView3D::useMotionQuality}
\vspace{-0.65cm}
\hrulefill
\index{wxView3D!useMotionQuality}
\index{useMotionQuality!wxView3D}
\begin{flushleft}
\texttt{void useMotionQuality();}
\end{flushleft}



\subsubsection*{wxView3D::changeRenderingQuality}
\vspace{-0.65cm}
\hrulefill
\index{wxView3D!changeRenderingQuality}
\index{changeRenderingQuality!wxView3D}
\begin{flushleft}
\texttt{void changeRenderingQuality(GLRenderer::RenderingQuality q);}
\end{flushleft}



\subsubsection*{wxView3D::getRenderingQuality}
\vspace{-0.65cm}
\hrulefill
\index{wxView3D!getRenderingQuality}
\index{getRenderingQuality!wxView3D}
\begin{flushleft}
\texttt{GLRenderer::RenderingQuality getRenderingQuality();}
\end{flushleft}



\subsubsection*{wxView3D::changeMotionQuality}
\vspace{-0.65cm}
\hrulefill
\index{wxView3D!changeMotionQuality}
\index{changeMotionQuality!wxView3D}
\begin{flushleft}
\texttt{void changeMotionQuality(GLRenderer::RenderingQuality q);}
\end{flushleft}



\subsubsection*{wxView3D::getMotionQuality}
\vspace{-0.65cm}
\hrulefill
\index{wxView3D!getMotionQuality}
\index{getMotionQuality!wxView3D}
\begin{flushleft}
\texttt{GLRenderer::RenderingQuality getMotionQuality();}
\end{flushleft}



\subsubsection*{wxView3D::getMouseEvent}
\vspace{-0.65cm}
\hrulefill
\index{wxView3D!getMouseEvent}
\index{getMouseEvent!wxView3D}
\begin{flushleft}
\texttt{wxMouseEvent getMouseEvent();}
\end{flushleft}



\subsubsection*{wxView3D::adoptMouseCommand}
\vspace{-0.65cm}
\hrulefill
\index{wxView3D!adoptMouseCommand}
\index{adoptMouseCommand!wxView3D}
\begin{flushleft}
\texttt{void adoptMouseCommand(wxViewCommand* c);}
\end{flushleft}



\subsubsection*{wxView3D::orphanMouseCommand}
\vspace{-0.65cm}
\hrulefill
\index{wxView3D!orphanMouseCommand}
\index{orphanMouseCommand!wxView3D}
\begin{flushleft}
\texttt{wxViewCommand *orphanMouseCommand();}
\end{flushleft}



\subsubsection*{wxView3D::getRenderer}
\vspace{-0.65cm}
\hrulefill
\index{wxView3D!getRenderer}
\index{getRenderer!wxView3D}
\begin{flushleft}
\texttt{GLRenderer *getRenderer();}
\end{flushleft}



\printindex
\end{document}
